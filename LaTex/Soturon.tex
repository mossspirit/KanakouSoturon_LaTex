\input soturon16.tex
\bibliographystyle{junsrt}

\begin{document}


\Abstract{
    ここでは簡単に本研究の説明を行います.
}

\SetFront

%1
\chapter{はじめに}
%1.1
\section{研究背景}
\label{sec:1.1}
これはLaTeXの雛形です.章の名前などは適宜変更,追加してください.

%1.2
\section{研究目的}
\ref{sec:1.1}で述べたように節の名前も同様です.

%2
\chapter{画像の挿入}
ここでは図の追加をします.
図はpdfで追加します.
\begin{figure}[hb]
    \begin{center}  
        \includegraphics[width=0.725\linewidth]{Figure/kanagawa.pdf} %幅の倍率で設定すると調整しやすい.
        \caption{神奈川工科大学\cite{kanakouPage}}
        \label{fig:kanakou}
    \end{center}
\end{figure}


%3
\chapter{仕様する技術}
\label{sec:3}
表は表\ref{tab:Romazi}のように追加します.

\begin{table}[htbp]
    \centering
    \caption{ローマ字表です}
    \label{tab:Romazi}
    \begin{tabular}{l|cc}
      & a & k  \\ \hline
    a & a & Ka \\
    i & i & Ki \\
    u & u & Ku
    \end{tabular}
\end{table}

%4
\chapter{システムの概要}
\label{sec:4}
%直接ソースコードを書く場合
\begin{lstlisting}[caption=UUIDの変更関数,label=ChangeUUID, firstnumber=1]
    #include<stdio.h>
    void main(){
        printf("TeXファイルに直接ソースコードを書く場合です.")
    }
\end{lstlisting}


%5
\chapter{むすび}
LaTeXを使って簡単に卒論を書こう!!


%参考文献
\clearpage
\addcontentsline{toc}{chapter}{参考文献}
\bibliography{article}

\Appendix
\chapter{プログラムリスト}
本プログラムを記す.
\lstinputlisting[caption=SoturonTex,label=soturon]{Soturon.tex}
\end{document}